\documentclass{article}

\usepackage[margin=1.0in]{geometry}
\usepackage{amsmath}
\usepackage{lmodern}

\begin{document}
\section{Linearization}
We begin with equations (15) and (16) in HKT,
\begin{equation}
    \label{eq:hk15}
    \frac{\partial \psi}{\partial t} + \vec{V}\cdot \vec{\nabla} \psi= \frac{\eta}{\mu_0} \nabla ^2 \psi
\end{equation} 
\begin{equation}
    \label{eq:hk16}
    \rho\left( \frac{\partial }{\partial t} + \vec{V}\cdot \vec{\nabla} \right)\omega_z = \vec{B}\cdot \vec{\nabla}j_z
\end{equation} 

where
\begin{equation}
    \vec{V}= \hat{z}\times \vec{\nabla}\phi
\end{equation} 

\begin{equation}
    \omega_z = \hat{z}\cdot \vec{\nabla}\times \vec{V}=\nabla ^{2}\phi
\end{equation} 

Perturbation Methods approximate a solution from an exact solution from a simpler problem by introducing a small
``perturbing'' parameter $\varepsilon$ and express a solution in the form of a series,
\begin{equation}
    \psi=\sum_{n=0}^{\infty} \varepsilon^{n}\psi_n=\psi_0+\varepsilon\psi_1+\varepsilon^{2}\psi_2+\ldots
\end{equation} 

For the project, we are interested in the 0th and 1st order solutions. This means including terms up to (and including) $\sim O(\varepsilon^2)$ are
enough to reveal the 0th and 1st order solutions. Expanding (\ref{eq:hk15}) and (\ref{eq:hk16}), we get
\begin{equation}
    \frac{\partial}{\partial t} \left(\psi_0+\varepsilon\psi_1+\varepsilon^{2}\psi_2 \right) + \left( \vec{V}_0 + \varepsilon \vec{V}_1 + \varepsilon^2 \vec{V} \right) \cdot \vec{\nabla} \left(\psi_0+\varepsilon\psi_1+\varepsilon^{2}\psi_2 \right) = \frac{\eta}{\mu_0}\nabla ^{2} \left( \psi_0+\varepsilon\psi_1+\varepsilon^{2}\psi_2 \right) 
\end{equation} 

\begin{equation}
    \begin{split}
        \rho&\left( \frac{\partial }{\partial t} + \left( \vec{V}_0 + \varepsilon \vec{V}_1 + \varepsilon^2 \vec{V}_2 \right) \cdot \vec{\nabla} \right) \left(\omega_z^{(0)} + \varepsilon \omega_z^{(1)} + \varepsilon ^{2} \omega_z^{(2)} \right) =\\
        &\left( \vec{B}_0 + \varepsilon \vec{B}_1 + \varepsilon^2 \vec{B}_2 \right) \cdot \vec{\nabla}\left(j_z^{(0)} + \varepsilon j_z^{(1)} + \varepsilon^2 j_z^{(2)}  \right) 
    \end{split}
\end{equation} 
where $\vec{V}_i = \hat{z}\times \vec{\nabla} \phi_i$ and $\omega_z^{(i)}=\nabla ^{2}\phi_i.$

Re-grouping the terms by order of $\varepsilon$, we get for (\ref{eq:hk15})
\begin{equation}
    \label{eq:hk15lin}
    \begin{split}
        \varepsilon^0 &: \frac{\partial \psi_0}{\partial t} + \vec{V}_0 \cdot \vec{\nabla} \psi_0 = \frac{\eta}{\mu_0}\nabla ^{2}\psi_0 \\
        \varepsilon^1 &: \frac{\partial \psi_1}{\partial t} + \vec{V}_0 \cdot \vec{\nabla} \psi_1 + \vec{V}_1 \cdot \vec{\nabla} \psi_0 = \frac{\eta}{\mu_0}\nabla ^{2}\psi_1
    \end{split}
\end{equation}

and we get for (\ref{eq:hk16})
\begin{equation}
    \label{eq:hk16lin}
    \begin{split}
        \varepsilon^0 &: \rho\left[ \frac{\partial}{\partial t} + \vec{V}_0 \cdot \vec{\nabla} \right] \omega_z^{(0)} = \vec{B}_0 \cdot \vec{\nabla}j_z^{(0)} \\
        \varepsilon^1 &: \rho\left[\frac{\partial \omega_z^{(1)}}{\partial t} + \vec{V}_0 \cdot \vec{\nabla}\omega_z^{(1)} + \vec{V}_1 \cdot \vec{\nabla}\omega_z^{(0)}\right] = \vec{B}_0 \cdot \vec{\nabla}j_z^{(1)} + \vec{B}_1 \cdot \vec{\nabla}j_z^{(0)}
    \end{split}
\end{equation} 

Additional assumptions can be made to further simplify the PDEs. First, since resistivity $\eta\sim O(\varepsilon)$, the resistivity term in the RH side of the 1st order equation in (\ref{eq:hk15lin}) is negligible (set to zero.)
Another assumption is that $\vec{V}_0=0$, since we are concerned with a system in equilibrium. For (\ref{eq:hk15lin}), the resulting equations become
\begin{equation}
    \label{eq:hk15lin2}
    \begin{split}
        \varepsilon^0 &: \frac{\partial \psi_0}{\partial t} = 0 \\
        \varepsilon^1 &: \frac{\partial \psi_1}{\partial t} + \vec{V}_1 \cdot \vec{\nabla} \psi_0 = 0
    \end{split}
\end{equation} 

and for (\ref{eq:hk16lin}),
\begin{equation}
    \label{eq:hk15lin2}
    \begin{split}
        \varepsilon^0 &: \rho\frac{\partial}{\partial t}\omega_z^{(0)} = \rho\frac{\partial}{\partial t}\nabla ^{2} \phi_0 = \vec{B}_0 \cdot \vec{\nabla}j_z^{(0)} = 0 \\ 
        \varepsilon^1 &: \rho\left[\frac{\partial \omega_z^{(1)}}{\partial t} + \vec{V}_1 \cdot \vec{\nabla}\omega_z^{(0)}\right] = \vec{B}_0 \cdot \vec{\nabla}j_z^{(1)} + \vec{B}_1 \cdot \vec{\nabla}j_z^{(0)}
    \end{split}
\end{equation} 
\subsection{Linearization for (\ref{eq:hk15})}
Evaluating the 1st order equation of (\ref{eq:hk15lin2}),
\begin{equation}
    \begin{split}
        \frac{\partial \psi_1}{\partial t} + \vec{V}_1 \cdot \vec{\nabla} \psi_0 &= \frac{\partial \psi_1}{\partial t} + \hat{z} \times \vec{\nabla} \phi_1 \cdot \vec{\nabla} \psi_0 = \\
                                                                                 &= \frac{\partial \psi_1}{\partial t} - \frac{\partial \psi_0}{\partial x} \frac{\partial \phi_1}{\partial y} + \frac{\partial \psi_0}{\partial y} \frac{\partial \phi_1}{\partial x} \\ 
    \end{split}
\end{equation} 

Let $\psi_0=\frac{B_0}{2a}x^2$, $\psi_1=\psi_1\left( x,t \right) \cos{ky}$ and $\phi_1=\phi_1\left( x,t \right) \sin{ky}$,
\begin{equation}
   \begin{split}
       \frac{\partial \psi_1}{\partial t} - \frac{\partial \psi_0}{\partial x} \frac{\partial \phi_1}{\partial y} + \frac{\partial \psi_0}{\partial y} \frac{\partial \phi_1}{\partial x} &= \\
                                                                                                                                                                                          &= \frac{\partial \psi_1}{\partial t} \cos{ky} - k\frac{\partial \psi}{\partial x} \phi_1\left( x,t \right) \cos{ky} \\
                                                                                                                                                                                          &= \frac{\partial \psi_1}{\partial t} -\frac{kB_0x}{a}\phi_1\left( x,t \right) = 0
   \end{split} 
\end{equation} 

\subsection{Linearization for (\ref{eq:hk16})}
By taking the curl of both sides of (14) in HKT, the force balance equation, we get
\begin{equation}
    \label{eq:hk14}
    \begin{split}
        \nabla \times \left[\rho \left( \frac{\partial}{\partial t} \vec{v} + \left( \vec{v} \cdot \vec{\nabla} \right) \cdot \vec{v} \right)\right] &= \nabla \times \left[- \nabla p + \vec{j}\times \vec{B}\right] \\
                                                                                                                                                     &= \nabla \times \left( \vec{j}\times \vec{B} \right) 
    \end{split}
\end{equation} 
Getting the linearized 1st order expression of the RH side of (\ref{eq:hk14}), we can use vector calculus identities to rearrange the expression into the following form,
\begin{equation}
    \begin{split}
        \nabla \times \left( \vec{j}_0 \times \vec{B}_1 \right) &+ \nabla \times \left( \vec{j}_1 + \vec{B}_0 \right) \\
                                                                &= \vec{j}_0 (\vec{\nabla} \cdot \vec{B}_1) - \vec{B}_1 (\vec{\nabla} \cdot \vec{j}_0) \\
                                                                &+ \left( \vec{B}_1 \cdot \vec{\nabla}  \right) \vec{j}_0 - \left( \vec{j}_0 \cdot  \vec{\nabla}  \right) \vec{B}_1 \\
                                                                &+ \vec{j}_1 (\vec{\nabla} \cdot \vec{B}_0) - \vec{B}_0 (\vec{\nabla} \cdot \vec{j}_1) \\
                                                                &+ \left( \vec{B}_0 \cdot \vec{\nabla}  \right) \vec{j}_1 - \left( \vec{j}_1 \cdot  \vec{\nabla}  \right) \vec{B}_0 \\
    \end{split}
\end{equation} 
By applying relations from Maxwell's equations (i.e. $\vec{\nabla} \cdot \vec{B}=0$) and the continuity of charge conditions due to equilibrium (i.e. $\vec{\nabla} \cdot \vec{j}=0$), we can eliminate the following terms,
\begin{equation}
    \begin{split}
        \nabla \times \left( \vec{j}_0 \times \vec{B}_1 \right) &+ \nabla \times \left( \vec{j}_1 + \vec{B}_0 \right) = \\
                                                                &+ \left( \vec{B}_1 \cdot \vec{\nabla}  \right) \vec{j}_0 - \left( \vec{j}_0 \cdot  \vec{\nabla}  \right) \vec{B}_1 \\
                                                                &+ \left( \vec{B}_0 \cdot \vec{\nabla}  \right) \vec{j}_1 - \left( \vec{j}_1 \cdot  \vec{\nabla}  \right) \vec{B}_0 \\
    \end{split}
\end{equation} 
Since we are concerned with the z-component of the expression, we can rewrite the RH side expression as
\begin{equation} \vec{B}_1 \cdot \vec{\nabla} j_z^{(0)} + \vec{B}_0 \cdot \vec{\nabla} j_z^{(1)} - \vec{j}_0 \cdot \vec{\nabla} B_z^{(1)} - \vec{j}_1 \cdot \vec{\nabla} B_z^{(0)}
\end{equation} 
From $\frac{\vec{\nabla} \times \vec{B}_i}{\mu_0}=\vec{j}_i$ and the fact that $B_z=\text{const.}$,
\begin{equation}
    \begin{split}
        \vec{B}_1 \cdot \vec{\nabla} j_z^{(0)} + \vec{B}_0 \cdot \vec{\nabla} j_z^{(1)} &= \vec{B}_1 \cdot \nabla \left( \frac{\vec{\nabla} \times \hat{z} \times \vec{\nabla}\psi_0}{\mu_0} \right) \\
                                                                                        &+ \vec{B}_0 \cdot \nabla \left( \frac{\vec{\nabla} \times \hat{z} \times \vec{\nabla}\psi_1}{\mu_0} \right) \\
                                                                                        &= \frac{1}{\mu_0} \left( \vec{B}_1 \cdot \nabla \left( \nabla^2 \psi_0 \right) + \vec{B}_0 \cdot \nabla \left( \nabla^2 \psi_1 \right) \right)
    \end{split}
\end{equation} 

which becomes
\begin{equation}
    \begin{split}
        \rho \frac{\partial}{\partial t} \nabla ^2 \phi_1 &= \frac{1}{\mu_0} \left( \vec{B}_1 \cdot \nabla \left( \nabla^2 \psi_0 \right) + \vec{B}_0 \cdot \nabla \left( \nabla^2 \psi_1 \right) \right) \\
        \rho \frac{\partial}{\partial t} \nabla ^2 \phi_1 &= \frac{1}{\mu_0} \left[ \frac{\partial \psi_1}{\partial x} \frac{\partial}{\partial y} \nabla ^2 \psi_0 - \frac{\partial \psi_1}{\partial y} \frac{\partial}{\partial x} \nabla ^2 \psi_0 + \frac{\partial \psi_0}{\partial x} \frac{\partial}{\partial y} \nabla ^2 \psi_1 - \frac{\partial \psi_0}{\partial y} \frac{\partial}{\partial x} \nabla ^2 \psi_1 \right]
    \end{split}
\end{equation} 

According to the paper, let $\psi_0=\frac{B_0}{2a}x^2$, $\psi_1=\psi_1\left( x,t \right) \cos{ky}$, and $\phi_1=\phi_1\left( x,t \right) \cos{ky}$. By inspection, the equation simplifies to
\begin{equation}
    \begin{split}
        \rho \frac{\partial}{\partial t} \nabla ^2 \phi_1 &= \frac{1}{\mu_0} \left[ \frac{\partial \psi_0}{\partial x} \frac{\partial}{\partial y} \nabla ^2 \psi_1 \right] = \frac{1}{\mu_0} \frac{\partial \psi_0}{\partial x} \left[ k \frac{\partial^2 \psi_1\left( x,t \right) }{\partial x^2} \cos{ky} - k^{3} \psi_1\left( x,t \right) \cos{ky} \right] \\
        \rho \frac{\partial}{\partial t} \nabla ^2 \phi_1(x,t)\cos{ky} &= -\frac{kB_0}{\mu_0 a} x \left[ \frac{\partial^2 \psi_1\left( x,t \right) }{\partial x^2} \cos{ky} - k^{2} \psi_1\left( x,t \right) \cos{ky} \right]
    \end{split}
\end{equation} 
Due to the properties of boundary layers, terms $\sim O(k^2)$ are negligible,
\begin{equation}
    \rho \frac{\partial}{\partial t} \nabla ^2 \phi_1(x,t)\cos{ky} = -\frac{kB_0}{\mu_0 a} x \left[ \frac{\partial^2 \psi_1\left( x,t \right) }{\partial x} \cos{ky} \right]
\end{equation} 
Introducing non-dimensional terms $\tau=\frac{t}{\tau_A}$, $\chi=\frac{x}{a}$,
\begin{equation}
    \begin{split}
        \rho \frac{\partial}{\partial t} \nabla^2 \phi &= -\frac{kB_0 \chi}{\mu_0 a^2} \left( \frac{\partial ^2}{\partial \chi^2} \psi_1 \right) \\
        \rho \frac{\partial}{\partial t} \left[ \frac{1}{a^2} \frac{\partial^2}{\partial \chi^2} \phi_1(x,t) \right] &= -\frac{kB_0 \chi}{\mu_0 a^2} \left( \frac{\partial ^2}{\partial \chi^2} \psi_1 \right) \\
        \rho \frac{\partial}{\partial t} \frac{\partial^2}{\partial \chi^2} \phi_1(x,t) &= -\frac{kB_0 \chi}{\mu_0} \left( \frac{\partial ^2}{\partial \chi^2} \psi_1 \right)
    \end{split}
\end{equation} 
\end{document}

