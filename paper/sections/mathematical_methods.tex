\documentclass[../main.tex]{subfiles}
\graphicspath{{\subfix{../images/}}}

\begin{document}
% \subsection{Overview, Motivation}
% This section may instead be placed in the appendix. Mathematical theories belong there.

\subsection{Boundary Layer Theory}

Boundary Layer Theory is a method to find approximate solutions to a differential equation, especially when they have a perturbing parameter such as $\varepsilon$. The theory also defines 
\textit{inner} and \textit{outer} solutions. Inner solutions represent regions of rapid change in the solution as $\varepsilon$ is varied, while outer solutions represent regions of slow change. This method pertains to the project as we are interested in MHD equilibrium (i.e. force balance in a plasma) at small values of resistivity (denoted by $\eta$.) The small resistivity parameter can be analogous to the small parameter $\varepsilon.$

\subsection{Perturbation Methods}

\subsection{Fourier Analysis}


\end{document}
