\documentclass[../main.tex]{subfiles}
\graphicspath{{\subfix{../images/}}}

\begin{document}
Fusion plasmas exhibit non-ideal characteristics. In ideal plasmas, where resistivity is zero, the magnetic flux is ``frozen-in,'' which means its topology cannot change over time. This means that magnetic flux lines cannot form loops in ideal plasma. This is not the case for fusion plasmas. To address this deviation from ideal MHD, a ``Multi-region Relaxed'' (MRx) model is used instead. MRx models propose that a toroidal cross section is made up of alternating layers of ideal plasma undergoing Taylor relaxation, a process of plasma transitioning into a lower energy state, occurs only in sub-regions of plasma \cite{dewar2017}. The thin layers are assumed to have zero thickness.

This project aims to test the robustness of ideal interfaces in MRxMHD. This entails testing the presence of interfaces in resistive plasmas. Typically, we expect current sheet structures to break down in such environments. Over the course of the project, we applied mathematical methods to partial differentiation equations presented in several papers. Such mathematical methods include boundary layer theory, perturbation methods, and fourier analysis. Additionally, the literature examined during the project includes papers from Dewar et al. (2017), Wang and Bhattacharjee (1992), and Hahm and Kulsrud (1985).

% Considerations
% Commentary more on resistivity (ex. resistivity magnitude)?
\end{document}
